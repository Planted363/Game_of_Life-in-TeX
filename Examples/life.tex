%%
%% # Conway's Game of Life - a (finite) implementation in TeX. #
%%



%% ## Setup. ##
\catcode`@=11\relax
\newcount\i \newcount\j \newcount\k \newcount\t \newcount\p \newcount\h \newcount\lb@size@x \newcount\lb@size@y \newcount\lb@iter \newcount\lb@diff
         \i=1        \j=1        \k=1        \t=1        \p=1        \h=1        \lb@size@x=47       \lb@size@y=60       \lb@iter=0        \lb@diff=0\relax

%% ## Title. ##
%% Here we set up the fonts.
\font\magtensc=cmcsc10 scaled 1500
\font\magtenrm=cmr10 scaled 2000
\font\magtenit=cmti10 scaled 1600
\baselineskip=1000pt \lineskip=2pt \lineskiplimit=16383pt \parindent=0pt \let\folio\relax

%% The title.
\def\title{%
    \line{\hskip -2em\magtensc Conway's \hss \tt Iteration \the\lb@iter}\vskip 5pt
    \line{\magtenrm Game of Life \hss \magtenit An implementation in \TeX\rlap{.}}\vskip 15pt
}

%% Loops over the entire board.
\def\lb@loop#1#2{%
    \i=1\j=1
    \loop
        {\loop#1\advance\j1\ifnum\j<\lb@size@x\repeat}#2%
        \advance\i1\relax
    \ifnum\i<\lb@size@y\repeat
}

%% The cells for alive and dead states.
\newbox\lc@deadbox
\setbox\lc@deadbox=\hbox{\hskip 1pt\vrule height 8pt width .1pt\leaders\hrule height .1pt\hskip 7.8pt\llap{\raise 7.9pt\hbox{\leaders\hrule height .1pt\hskip 7.8pt}}\vrule height 8pt width .1pt\hskip 1pt\relax}
\newbox\lc@alivebox
\setbox\lc@alivebox=\hbox{\hskip 1pt\vrule height 8pt width 8pt\hskip 1pt}
\def\lc@dead{\unhcopy\lc@deadbox}
\def\lc@alive{\unhcopy\lc@alivebox}

%% Loops over board creating empty (dead) cells.
\def\lb@init{%
    \lb@loop{\global\expandafter\let\csname lc@\the\i;\the\j\endcsname\lc@dead}{}%
}

%% Display board.
\def\lb@print{\lb@loop{\csname lc@\the\i;\the\j\endcsname}{\endgraf}}
\def\lb@print@next{%
    \lb@diff=0
    \lb@loop{%
        \expandafter\ifx\csname lc@\the\i;\the\j\expandafter\endcsname\csname lc@next@\the\i;\the\j\endcsname
        \else
            \global\advance\lb@diff1
        \fi
        \global\expandafter\let\csname lc@\the\i;\the\j\expandafter\endcsname\csname lc@next@\the\i;\the\j\endcsname
        \csname lc@\the\i;\the\j\endcsname}{\endgraf}%
    \ifnum\lb@diff=0
        \lb@stop
    \fi
}

%% ## Update processing. ##
\def\lb@tick{\relax\lb@loop{\lc@update}{}}
\def\lc@update{%
    \expandafter\ifx\csname lc@\the\i;\the\j\endcsname\lc@alive
        \lc@update@alive
    \else
        \lc@update@dead
    \fi
}

%% Offset the "cursor" by specified coordinates, checking if an alive cell exists.
\def\lc@check@numalive@peek#1#2{%
    \begingroup
        \advance\i#1\advance\j#2
        \expandafter\ifx\csname lc@\the\i;\the\j\endcsname\lc@alive
            \global\advance\k1\relax
        \fi
    \endgroup
}
%% Count the number of neighbours alive.
\def\lc@check@numalive{%
    \k=0\relax
    \lc@check@numalive@peek{1}{0}%
    \lc@check@numalive@peek{1}{1}%
    \lc@check@numalive@peek{0}{1}%
    \lc@check@numalive@peek{-1}{1}%
    \lc@check@numalive@peek{-1}{0}%
    \lc@check@numalive@peek{-1}{-1}%
    \lc@check@numalive@peek{0}{-1}%
    \lc@check@numalive@peek{1}{-1}%
}

\def\lc@update@alive{%
    \lc@check@numalive
    \ifnum\k=2
        \lc@update@to@alive
    \else
        \ifnum\k=3
            \lc@update@to@alive
        \else
            \lc@update@to@dead
        \fi
    \fi
}
\def\lc@update@dead{%
    \lc@check@numalive
    \ifnum\k=3
        \lc@update@to@alive
    \else
        \lc@update@to@dead
    \fi
}
\def\lc@update@to@dead{\global\expandafter\let\csname lc@next@\the\i;\the\j\endcsname\lc@dead}
\def\lc@update@to@alive{\global\expandafter\let\csname lc@next@\the\i;\the\j\endcsname\lc@alive}

%% ## Processing Control. ##
%% User command for forwarding arbitrary steps.
\def\lb@forw@once{\vfill\eject\advance\lb@iter1\title\lb@tick\lb@print@next}
\let\lb@forw@once@\lb@forw@once
\def\gameforw#1{%
    \h=1\begingroup
        \aftergroup\def\aftergroup\next\aftergroup{%
        \loop\aftergroup\lb@forw@once\advance\h1\ifnum\h<#1\repeat
        \aftergroup}%
    \endgroup\next
}

%% Halts processing.
\def\lb@stop{%
    \errmessage{All life ceased after iteration \the\lb@iter.}
    \let\lb@forw@once@\lb@forw@once\let\lb@forw@once\empty
}

%% ## Input handling. ##
%% Check for optional argument.
\def\game{\title\futurelet\next\x@game}
\def\x@game{%
    \ifx[\next
        \expandafter\@game
    \else
        \expandafter\@@game
    \fi
}
\def\@game[#1,#2]#3{%
    \let\lb@forw@once\lb@forw@once@
    \lb@iter=0\lb@size@x=#1\lb@size@y=#2%
    \lb@init\relax\game@parse#3,@END,%
}
\def\@@game#1{\@game[47,60]{#1}}

%% Parse a list of input coordinates separated by commas.
\def\game@parse@end{@END}
\def\game@parse#1,{%
    \def\next{#1}%
    \ifx\game@parse@end\next
        \let\next\lb@print
    \else
        \lb@conv#1/
        \let\next\game@parse
    \fi\next
}
%% Convert from user input (alphabet) to numbers.
\def\lb@conv#1#2/{%
    \p=`#2\advance\p-96\relax
    \t=`#1\advance\t-96\relax
    \lb@place\p/\t/
}
%% Place (make alive) cell at input coordinate.
\def\lb@place#1/#2/{\begingroup
    \i=#1\j=#2
    \global\expandafter\let\csname lc@\the\i;\the\j\endcsname\lc@alive
\endgroup}

\catcode`@=12\relax

\game{zz,zy,zx,yx,xz,xy,wy,vz,vw,vv,wv,xv,zv}
\gameforw{417}

\bye
